%!Mode:: "TeX:UTF-8"
%%% Local Variables:
%%% mode: latex
%%% TeX-master: t
%%% End:

\chapter{表格与算法}
\label{cha:6}

如果要调字号的话,直接用 \textbackslash begin\{footnotesize\} 跟 \textbackslash end\{footnotesize\}
那套东西不行。表 \ref{tb:example} 的是一个可行的方案,用不用由你。


\begin{table}
    \centering
    \caption{表格样例}
    \label{tb:example}
    \begin{tabular}{>{\wuhao}c|>{\xiaowu}c|>{\xiaowu}c|>{\xiaowu}c|>{\xiaowu}c|>{\xiaowu}c|>{\xiaowu}c|>{\xiaowu}c|>{\xiaowu}c|>{\xiaowu}c|>{\xiaowu}c}  \hline
    no1 & \multicolumn{2}{>{\xiaowu}c|}{$10^{-5}$} & \multicolumn{2}{>{\xiaowu}c|}{$2 \times 10^{-5}$} & \multicolumn{2}{>{\xiaowu}c|}{$3 \times 10^{-5}$} & \multicolumn{2}{>{\xiaowu}c|}{$4 \times 10^{-5}$} & \multicolumn{2}{>{\xiaowu}c}{$5 \times 10^{-5}$}
     \\ \hline
    something & $0$ & $0$ & $0\%$ & $0$ & $0\%$ & 0 & $0\%$ & 0 & $0\%$ & $0$
    \\ \hline
    \parbox[c][1cm][c]{2cm}{\centering 换行\\换行} & $111$ & $0$ & $111$ & $111$ & $111$ & $0\%$ & $1\%$ & $100\%$ & $2\%$ & $100\%$
    \\ \hline
    \end{tabular}
\end{table}


算法的话我用的是 algorithmicx,它可以自动处理算法编号,包括章节。main.tex 里面已经配好了,可以直接拿来用。
如果你不想用这个,请删掉 main.tex 的相关代码并加入自己的东西。算法 \ref{alg:example} 是一个示例。


\begin{algorithm}[htb!]  % 正常就h就好了,这里是为了放到一页上好看点
\caption{算法示例}
\label{alg:example}
\begin{algorithmic}[1]
\INPUT
    \Statex 输入1;
    \Statex 输入2;
\OUTPUT
    \Statex 输出1;
\INIT
    \Statex 初始化1;
    \State 步骤1; \Comment{还可以加注释哦}
    \For{$~i=2$~ {\bf to} $~N-1$~ }
        \State do something
    \EndFor
    \Repeat
    \State lalala
        \If{$a=0$}
            \State $b=0$
        \EndIf
    \Until{some condition}
    \State 步骤2。
\end{algorithmic}
\end{algorithm}


%%%%%%


\chapter{其他}
\label{chp:others}

没人规定一个文件里只能有一章,你想吧所有正文都放一个文件都没问题呀

\section{关于文字空格}

\subsection{dummy}

\subsubsection{这一级的标题不会进入目录的哦}

pdflatex 加上这个模板中英文混编的时候我是觉得特别难看了,比方说English这种,我这里是挤在一起的,数字如2000也是这样,公式如$x$也是这样,引用如\ref{chp:others}一样。

我是不喜欢这样啦,这点 Word 就做得好呀,不知 xelatex 是不是这样,或者跟模板有关啥的。
为了解决这个问题,我都加了空格。
然后,又感觉英文和中文标点之间不应该有空格呀,如果句首是英文不应该在前面有空格呀啥的,可麻烦了。

{\hei \sanhao 所以不是实在看不下去就不要加空格了吧。。。}

如果一定要加,还是比较推荐写的时候就手动加上,像我这样查找替换为主还是挺容易出错的。

我在写论文和查找替换的时候,采取了以下规则,可能稍微方便一点,但总是有漏网之鱼或者错误替换的情况出现,这时就得靠眼睛看了。

在写论文的时候注意:
\begin{enumerate}[1. ]
    \item 纯文字的话自己填上空格,如年份,中英文夹杂啥的(其实是我不会弄)。注意如果句首是英文的话,不要加空格,更不要加换行,因为一个换行也会被 tex 认为是空格;
    \item 不使用 \$\$ 写公式,统一用 align 就好了;
    \item 中文和 \$ 之间先不加空格,\textbackslash ref\{\} 前后先不加空格,之后统一替换。
\end{enumerate}

写完以后,顺序进行如表 \ref{tb:replace} 所示的替换。
{\hei 替换之前请好做好备份}。
我这里可是没做检查,不要同一个操作进行多次哦。
里面有普通的文字替换,也有正则表达式的替换,正则表达式替换可以用 sublime 啊,Notepad++ 什么的做。
其中有一个步骤是看你的公式用的是中文括号(前后不应加空格)还是英文括号(前后应该加空格),请只选做其中的一步。
注意直接在表里复制的话,粘贴出来有问题,所以请打开源代码,复制在 \verb+\verb|+ 和 \verb+|+ 之间的代码。

\begin{table}
    \centering
\caption{替换流程}
\label{tb:replace}
    \begin{tabular}{>{\xiaowu}c|>{\xiaowu}c|>{\xiaowu}c|>{\xiaowu}c|>{\xiaowu}c}  \hline
    针对人群 & 模式 & from & to  & 作用 \\\hline
    所有 & 普通 & {\$} & \$$\sim$  & 给所有行内公式加空格 \\\hline
    所有 & 正则 & \verb|(\\ref\{[^\}]*\})| & \verb*| \1 |  & 所有 ref 前后加空格 \\\hline
    所有 & 正则 & \verb*|(\\ref\{[^\}]*\}) (\([a-z]\))| & \verb*|\1\2 |  & \parbox[c][1cm][c]{4cm}{\centering 把刚才替换结果中像\\ “图 1.1~(a)”变成“图 1.1(a) ”} \\\hline
    仅中文括号 & 正则 & \verb*|( (\\ref\{[^\}]*\}) )| & \verb|(\1)|  & \parbox[c][1cm][c]{4cm}{\centering 删掉包含在中文括号中\\ ref 左右的空格} \\\hline
    \parbox[c][0.6cm][c]{1cm}{\centering 仅英文\\括号} & 正则 & \verb*|\( (\\ref\{[^\}]*\}) \)| & \verb*| (\1) |  & \parbox[c][1cm][c]{5cm}{\centering 把英文括号中 ref 左右\\ 的空格挪到括号外} \\\hline
    所有 & 正则 & \verb*|[ ~]([,。(;)])| & \verb|\1\2 |  & 删除标点前面的空格 \\\hline
    所有 & 正则 & \verb*|([,。(;)]\$)[ \~]| & \verb|\1\2 |  & \parbox[c][1cm][c]{5cm}{\centering 若标点后直接接公式,\\ 删除标点后面的空格}    \\ \hline
    \end{tabular}
\end{table}

\section{关于本模板附带的两个脚本文件}

本模板附带了两个 bash 脚本文件,一个是 git 一键提交(c.sh),另外一个是 git 自动 pull 并编译(auto\_pull.sh)。

因为我在 MAC 下写论文但是在 WIN 下编译,这套东西就挺有用的了。
写了一点想看结果,直接 ./c.sh (后面可以加参数表示 commit 的说明),那边 WIN 下跑的 auto\_pull.sh 就会自动拉下来编译,CTEX 自带的那个编辑器还会自动刷新 PDF。

如果不想用 winedt 做编辑的话,我觉得还是挺方便的,顺便连备份都做了。


\section{如果你正在用或者想用 VIM 写论文}

本人基本等价于标准程序猿一枚。

如果你正在用或者想在 MAC 系统下用 VIM 写论文,并且你发现 Normal 模式经常被中文输入法打扰,你可以考虑这一页的解决方案\onlinecite{macvim}。虽然不完美,但总比没有好。

WIN 下应该也有类似的解决方案,但我没找过。

才不会告诉你我的毕业论文跟这个文档都是在 VIM 下码的字呢。

%%%

\chapter{结束语}
\label{chp:ending}

就这些了,祝你顺利毕业$\sim$



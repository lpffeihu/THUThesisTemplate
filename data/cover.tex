%!Mode:: "TeX:UTF-8"
%%% Local Variables:
%%% mode: latex
%%% TeX-master: t
%%% End:
\secretlevel{绝密} \secretyear{2100}

\ctitle{一份简明的THUThesis模板}
% 根据自己的情况选,不用这样复杂
\makeatletter
\ifthu@bachelor\relax\else
  \ifthu@doctor
    \cdegree{工学博士}
  \else
    \ifthu@master
      \cdegree{工学硕士}
    \fi
  \fi
\fi
\makeatother


\cdepartment[电子]{电子工程系}
\cmajor{信息与通信工程}
\cauthor{匿名君}
\csupervisor{某老师教授}
% 如果没有副指导老师或者联合指导老师,把下面两行相应的删除即可。
%\cassosupervisor{陈文光教授}
%\ccosupervisor{某某某教授}
% 日期自动生成,如果你要自己写就改这个cdate
%\cdate{\CJKdigits{\the\year}年\CJKnumber{\the\month}月}

% 博士后部分
% \cfirstdiscipline{计算机科学与技术}
% \cseconddiscipline{系统结构}
% \postdoctordate{2009年7月——2011年7月}

\etitle{A Brief Template for THUThusis}
% 这块比较复杂,需要分情况讨论:
% 1. 学术型硕士
%    \edegree:必须为Master of Arts或Master of Science(注意大小写)
%              “哲学、文学、历史学、法学、教育学、艺术学门类,公共管理学科
%               填写Master of Arts,其它填写Master of Science”
%    \emajor:“获得一级学科授权的学科填写一级学科名称,其它填写二级学科名称”
% 2. 专业型硕士
%    \edegree:“填写专业学位英文名称全称”
%    \emajor:“工程硕士填写工程领域,其它专业学位不填写此项”
% 3. 学术型博士
%    \edegree:Doctor of Philosophy(注意大小写)
%    \emajor:“获得一级学科授权的学科填写一级学科名称,其它填写二级学科名称”
% 4. 专业型博士
%    \edegree:“填写专业学位英文名称全称”
%    \emajor:不填写此项
\edegree{Doctor of Philosophy}
\emajor{Information and Communication Engineering}
\eauthor{Ni Mingjun}
\esupervisor{Professor Mou Laoshi}
%\eassosupervisor{Chen Wenguang}
% 这个日期也会自动生成,你要改么?
% \edate{December, 2005}

% 定义中英文摘要和关键字
\begin{cabstract}
本文是一份简要的~THUThisis~模板。

(博士)摘要要求是 500--800 字,和关键字必须放在一页内

\end{cabstract}

\ckeywords{论文;模板;关键字;不超过五个;紫薯补丁}

\begin{eabstract}

Translation for Chinese version, do not need to be in 1 page.

\end{eabstract}

\ekeywords{thesis; template; keyword; no more than five; patch for word count}

%!Mode:: "TeX:UTF-8"
%%% Local Variables:
%%% mode: latex
%%% TeX-master: t
%%% End:

\documentclass[doctor]{thuthesis}
% \documentclass[%
%   bachelor|master|doctor|postdoctor, % mandatory option
%   xetex|pdftex|dvips|dvipdfm, % optional
%   secret,
%   openany|openright,
%   arialtoc,arialtitle]{thuthesis}

% 所有其它可能用到的包都统一放到这里了,可以根据自己的实际添加或者删除。
\usepackage{thutils}
\usepackage{bm}
\usepackage{dsfont}


\usepackage{color}
\usepackage{algorithm,algpseudocode}



\DeclareMathOperator*{\argmin}{\arg\!\min}
\DeclareMathOperator*{\argmax}{\arg\!\max}

\newcommand{\tabincell}[2]{\begin{tabular}{@{}#1@{}}#2\end{tabular}}
\newcommand{\red}{\textcolor[rgb]{1.00,0.00,0.00}}
\newcommand{\blue}{\textcolor[rgb]{0.00,0.00,1.00}}

\DeclareMathOperator{\sgn}{sgn}

\floatname{algorithm}{算法}
\makeatletter
\@addtoreset{algorithm}{chapter}% algorithm counter resets every chapter
\makeatother
\renewcommand{\thealgorithm}{\thechapter.\arabic{algorithm}}% Algorithm # is <chapter>.<algorithm>
\algnewcommand\algorithmicinput{\textbf{输入:}}
\algnewcommand\INPUT{\item[\algorithmicinput]}
\algnewcommand\algorithmicoutput{\textbf{输出:}}
\algnewcommand\OUTPUT{\item[\algorithmicoutput]}
\algnewcommand\algorithmicinit{\textbf{初始化:}}
\algnewcommand\INIT{\item[\algorithmicinit]}


% 你可以在这里修改配置文件中的定义,导言区可以使用中文。
% \def\myname{薛瑞尼}

\begin{document}

% 定义所有的eps文件在 figures 子目录下
\graphicspath{{figures/}}


%%% 封面部分
\frontmatter
%!Mode:: "TeX:UTF-8"
%%% Local Variables:
%%% mode: latex
%%% TeX-master: t
%%% End:
\secretlevel{绝密} \secretyear{2100}

\ctitle{一份简明的THUThesis模板}
% 根据自己的情况选,不用这样复杂
\makeatletter
\ifthu@bachelor\relax\else
  \ifthu@doctor
    \cdegree{工学博士}
  \else
    \ifthu@master
      \cdegree{工学硕士}
    \fi
  \fi
\fi
\makeatother


\cdepartment[电子]{电子工程系}
\cmajor{信息与通信工程}
\cauthor{匿名君}
\csupervisor{某老师教授}
% 如果没有副指导老师或者联合指导老师,把下面两行相应的删除即可。
%\cassosupervisor{陈文光教授}
%\ccosupervisor{某某某教授}
% 日期自动生成,如果你要自己写就改这个cdate
%\cdate{\CJKdigits{\the\year}年\CJKnumber{\the\month}月}

% 博士后部分
% \cfirstdiscipline{计算机科学与技术}
% \cseconddiscipline{系统结构}
% \postdoctordate{2009年7月——2011年7月}

\etitle{A Brief Template for THUThusis}
% 这块比较复杂,需要分情况讨论:
% 1. 学术型硕士
%    \edegree:必须为Master of Arts或Master of Science(注意大小写)
%              “哲学、文学、历史学、法学、教育学、艺术学门类,公共管理学科
%               填写Master of Arts,其它填写Master of Science”
%    \emajor:“获得一级学科授权的学科填写一级学科名称,其它填写二级学科名称”
% 2. 专业型硕士
%    \edegree:“填写专业学位英文名称全称”
%    \emajor:“工程硕士填写工程领域,其它专业学位不填写此项”
% 3. 学术型博士
%    \edegree:Doctor of Philosophy(注意大小写)
%    \emajor:“获得一级学科授权的学科填写一级学科名称,其它填写二级学科名称”
% 4. 专业型博士
%    \edegree:“填写专业学位英文名称全称”
%    \emajor:不填写此项
\edegree{Doctor of Philosophy}
\emajor{Information and Communication Engineering}
\eauthor{Ni Mingjun}
\esupervisor{Professor Mou Laoshi}
%\eassosupervisor{Chen Wenguang}
% 这个日期也会自动生成,你要改么?
% \edate{December, 2005}

% 定义中英文摘要和关键字
\begin{cabstract}
本文是一份简要的~THUThisis~模板。

(博士)摘要要求是 500--800 字,和关键字必须放在一页内

\end{cabstract}

\ckeywords{论文;模板;关键字;不超过五个;紫薯补丁}

\begin{eabstract}

Translation for Chinese version, do not need to be in 1 page.

\end{eabstract}

\ekeywords{thesis; template; keyword; no more than five; patch for word count}

\makecover

% 目录
\tableofcontents

% 符号对照表
%!Mode:: "TeX:UTF-8"
\begin{denotation}
\item[DSP] 数字信号处理(Digital Signal Processing)
\item[$\forall$] 任取

%\item[]

\end{denotation}



%%% 正文部分
\mainmatter
%!Mode:: "TeX:UTF-8"
%%% Local Variables:
%%% mode: latex
%%% TeX-master: t
%%% End:

\chapter{引言}

\label{cha:intro}

这篇文章介绍的主要是如何把~THUThesis~编通,以及一些注意事项。模板的进一步说明在本模板带的文件
ThuThesis\_User\_Guide\_3.0.pdf
中。

至于文章内容如何组织,请参考研究生院的说明\cite{shuoming}和师兄的论文。

至于那些随便找个模板就能编过的,咱得膜拜一下。

我是 2016 年 1 月毕业的,至少在这个时间点上该模板还有效。



%!Mode:: "TeX:UTF-8"
%%% Local Variables:
%%% mode: latex
%%% TeX-master: t
%%% End:

\chapter{编译注意事项}
\label{cha:compile}

\begin{enumerate}[1. ]
    \item 编译用 CTeX\_2.9.2.164\_Full 版本中的 pdflatex,北邮人上有,不要做更新。如果更新了编译不过就重装一下吧。如果有人搞通了 xelatex,教教我哈$\sim$;
    \item 编译正文打开 main.tex 并编译;编译书脊打开 shuji.tex 并编译;
    \item 各章节放在 data/ 下面,main.tex 中有引用,可以增删;
    \item 图放在 figures/ 下面;
    \item bib在 ref/refs.bib 文件中;
    \item 如果文章没有任何引用 \textbackslash cite 或者 \textbackslash onlinecite,编译可能会报错;
    \item 我有时会遇到编译报错一次后,再编译出很奇怪的错误的情况。如果是 main.tex 中报的错,删掉 main.out 重新编译,再不行删掉 main.aux;如果是 chapxx.tex 报的错,删掉 chapxx.aux 再重新编译即可。
\end{enumerate}







%!Mode:: "TeX:UTF-8"
%%% Local Variables:
%%% mode: latex
%%% TeX-master: t
%%% End:

\chapter{公式}
\label{cha:3}

equation 环境+\textbackslash nonumber 时会跳号(虽然编号没显示,但实际上编号+1),像这样。


\begin{equation}
{\text{It SEEMS good here.}} \nonumber
\end{equation}

\begin{equation}
{\text{But look at my equation number!!!}}
\end{equation}

\begin{equation*}
{\text{equation* seems OK.}} \nonumber
\end{equation*}

\begin{equation}
{\text{My equation number is good!!!}}
\end{equation}






遇到此问题可用align环境(貌似所有equation都可替换成align,最后做个查找替换也行)。


\begin{align}
{\text{It SEEMS good here.}} \nonumber
\end{align}

\begin{align}
{\text{And is really OK.}}
\end{align}




如果你没遇到这个问题,说明也许是我编译器的问题,或者人品问题?!



%!Mode:: "TeX:UTF-8"
%%% Local Variables:
%%% mode: latex
%%% TeX-master: t
%%% End:

\chapter{定理定义环境和证明以及字号字体}
\label{cha:4}

定义定理啥的在 ThuThesis\_User\_Guide\_3.0.pdf 中有详尽的说明。


像这样。


\begin{theorem}
我是定理。
\end{theorem}

证明我是直接抄师兄的做法了。


\noindent{\hei 证明}:

balabalabala

\noindent{\hei 证毕}。


字号字体也在 ThuThesis\_User\_Guide\_3.0.pdf 说的很好了。
刚才证明和证毕几个字已经用到了黑体。



%!Mode:: "TeX:UTF-8"
%%% Local Variables:
%%% mode: latex
%%% TeX-master: t
%%% End:

\chapter{图片}
\label{cha:5}

之前用的 subplot 不行了。这里用 subfloat。如图 \ref{fig:example}。

\begin{figure}
\centering
\subfloat[可以有子标题也可以没有]{%
  \includegraphics[width=0.4\textwidth]{1.pdf}
}
\subfloat[]{%
  \includegraphics[width=0.4\textwidth]{2.pdf}
}
\\
\subfloat[I'm 3]{%
  \includegraphics[width=0.4\textwidth]{3.pdf}
}
\hfill
\subfloat[And at last]{%
  \includegraphics[width=0.4\textwidth]{4.pdf}
}
\caption{总标题}
\label{fig:example}
\end{figure}




%!Mode:: "TeX:UTF-8"
%%% Local Variables:
%%% mode: latex
%%% TeX-master: t
%%% End:

\chapter{表格与算法}
\label{cha:6}

如果要调字号的话,直接用 \textbackslash begin\{footnotesize\} 跟 \textbackslash end\{footnotesize\}
那套东西不行。表 \ref{tb:example} 的是一个可行的方案,用不用由你。


\begin{table}
    \centering
    \caption{表格样例}
    \label{tb:example}
    \begin{tabular}{>{\wuhao}c|>{\xiaowu}c|>{\xiaowu}c|>{\xiaowu}c|>{\xiaowu}c|>{\xiaowu}c|>{\xiaowu}c|>{\xiaowu}c|>{\xiaowu}c|>{\xiaowu}c|>{\xiaowu}c}  \hline
    no1 & \multicolumn{2}{>{\xiaowu}c|}{$10^{-5}$} & \multicolumn{2}{>{\xiaowu}c|}{$2 \times 10^{-5}$} & \multicolumn{2}{>{\xiaowu}c|}{$3 \times 10^{-5}$} & \multicolumn{2}{>{\xiaowu}c|}{$4 \times 10^{-5}$} & \multicolumn{2}{>{\xiaowu}c}{$5 \times 10^{-5}$}
     \\ \hline
    something & $0$ & $0$ & $0\%$ & $0$ & $0\%$ & 0 & $0\%$ & 0 & $0\%$ & $0$
    \\ \hline
    \parbox[c][1cm][c]{2cm}{\centering 换行\\换行} & $111$ & $0$ & $111$ & $111$ & $111$ & $0\%$ & $1\%$ & $100\%$ & $2\%$ & $100\%$
    \\ \hline
    \end{tabular}
\end{table}


算法的话我用的是 algorithmicx,它可以自动处理算法编号,包括章节。main.tex 里面已经配好了,可以直接拿来用。
如果你不想用这个,请删掉 main.tex 的相关代码并加入自己的东西。算法 \ref{alg:example} 是一个示例。


\begin{algorithm}[htb!]  % 正常就h就好了,这里是为了放到一页上好看点
\caption{算法示例}
\label{alg:example}
\begin{algorithmic}[1]
\INPUT
    \Statex 输入1;
    \Statex 输入2;
\OUTPUT
    \Statex 输出1;
\INIT
    \Statex 初始化1;
    \State 步骤1; \Comment{还可以加注释哦}
    \For{$~i=2$~ {\bf to} $~N-1$~ }
        \State do something
    \EndFor
    \Repeat
    \State lalala
        \If{$a=0$}
            \State $b=0$
        \EndIf
    \Until{some condition}
    \State 步骤2。
\end{algorithmic}
\end{algorithm}


%%%%%%


\chapter{其他}
\label{chp:others}

没人规定一个文件里只能有一章,你想吧所有正文都放一个文件都没问题呀

\section{关于文字空格}

\subsection{dummy}

\subsubsection{这一级的标题不会进入目录的哦}

pdflatex 加上这个模板中英文混编的时候我是觉得特别难看了,比方说English这种,我这里是挤在一起的,数字如2000也是这样,公式如$x$也是这样,引用如\ref{chp:others}一样。

我是不喜欢这样啦,这点 Word 就做得好呀,不知 xelatex 是不是这样,或者跟模板有关啥的。
为了解决这个问题,我都加了空格。
然后,又感觉英文和中文标点之间不应该有空格呀,如果句首是英文不应该在前面有空格呀啥的,可麻烦了。

{\hei \sanhao 所以不是实在看不下去就不要加空格了吧。。。}

如果一定要加,还是比较推荐写的时候就手动加上,像我这样查找替换为主还是挺容易出错的。

我在写论文和查找替换的时候,采取了以下规则,可能稍微方便一点,但总是有漏网之鱼或者错误替换的情况出现,这时就得靠眼睛看了。

在写论文的时候注意:
\begin{enumerate}[1. ]
    \item 纯文字的话自己填上空格,如年份,中英文夹杂啥的(其实是我不会弄)。注意如果句首是英文的话,不要加空格,更不要加换行,因为一个换行也会被 tex 认为是空格;
    \item 不使用 \$\$ 写公式,统一用 align 就好了;
    \item 中文和 \$ 之间先不加空格,\textbackslash ref\{\} 前后先不加空格,之后统一替换。
\end{enumerate}

写完以后,顺序进行如表 \ref{tb:replace} 所示的替换。
{\hei 替换之前请好做好备份}。
我这里可是没做检查,不要同一个操作进行多次哦。
里面有普通的文字替换,也有正则表达式的替换,正则表达式替换可以用 sublime 啊,Notepad++ 什么的做。
其中有一个步骤是看你的公式用的是中文括号(前后不应加空格)还是英文括号(前后应该加空格),请只选做其中的一步。
注意直接在表里复制的话,粘贴出来有问题,所以请打开源代码,复制在 \verb+\verb|+ 和 \verb+|+ 之间的代码。

\begin{table}
    \centering
\caption{替换流程}
\label{tb:replace}
    \begin{tabular}{>{\xiaowu}c|>{\xiaowu}c|>{\xiaowu}c|>{\xiaowu}c|>{\xiaowu}c}  \hline
    针对人群 & 模式 & from & to  & 作用 \\\hline
    所有 & 普通 & {\$} & \$$\sim$  & 给所有行内公式加空格 \\\hline
    所有 & 正则 & \verb|(\\ref\{[^\}]*\})| & \verb*| \1 |  & 所有 ref 前后加空格 \\\hline
    所有 & 正则 & \verb*|(\\ref\{[^\}]*\}) (\([a-z]\))| & \verb*|\1\2 |  & \parbox[c][1cm][c]{4cm}{\centering 把刚才替换结果中像\\ “图 1.1~(a)”变成“图 1.1(a) ”} \\\hline
    仅中文括号 & 正则 & \verb*|( (\\ref\{[^\}]*\}) )| & \verb|(\1)|  & \parbox[c][1cm][c]{4cm}{\centering 删掉包含在中文括号中\\ ref 左右的空格} \\\hline
    \parbox[c][0.6cm][c]{1cm}{\centering 仅英文\\括号} & 正则 & \verb*|\( (\\ref\{[^\}]*\}) \)| & \verb*| (\1) |  & \parbox[c][1cm][c]{5cm}{\centering 把英文括号中 ref 左右\\ 的空格挪到括号外} \\\hline
    所有 & 正则 & \verb*|[ ~]([,。(;)])| & \verb|\1\2 |  & 删除标点前面的空格 \\\hline
    所有 & 正则 & \verb*|([,。(;)]\$)[ \~]| & \verb|\1\2 |  & \parbox[c][1cm][c]{5cm}{\centering 若标点后直接接公式,\\ 删除标点后面的空格}    \\ \hline
    \end{tabular}
\end{table}

\section{关于本模板附带的两个脚本文件}

本模板附带了两个 bash 脚本文件,一个是 git 一键提交(c.sh),另外一个是 git 自动 pull 并编译(auto\_pull.sh)。

因为我在 MAC 下写论文但是在 WIN 下编译,这套东西就挺有用的了。
写了一点想看结果,直接 ./c.sh (后面可以加参数表示 commit 的说明),那边 WIN 下跑的 auto\_pull.sh 就会自动拉下来编译,CTEX 自带的那个编辑器还会自动刷新 PDF。

如果不想用 winedt 做编辑的话,我觉得还是挺方便的,顺便连备份都做了。


\section{如果你正在用或者想用 VIM 写论文}

本人基本等价于标准程序猿一枚。

如果你正在用或者想在 MAC 系统下用 VIM 写论文,并且你发现 Normal 模式经常被中文输入法打扰,你可以考虑这一页的解决方案\onlinecite{macvim}。虽然不完美,但总比没有好。

WIN 下应该也有类似的解决方案,但我没找过。

才不会告诉你我的毕业论文跟这个文档都是在 VIM 下码的字呢。

%%%

\chapter{结束语}
\label{chp:ending}

就这些了,祝你顺利毕业$\sim$




%%% 其它部分
\backmatter

% 本科生要这几个索引,研究生不要。选择性留下。
\makeatletter
\ifthu@bachelor
  % 插图索引
  \listoffigures
  % 表格索引
  \listoftables
  % 公式索引
  \listofequations
\fi
\makeatother


% 参考文献
\bibliographystyle{thubib}
\bibliography{ref/refs}


% 致谢

%%% Local Variables:
%%% mode: latex
%%% TeX-master: "../main"
%%% End:

\begin{ack}
%感谢党。

感谢老师。

感谢童鞋。

感谢家人。

\end{ack}


% 附录
%\begin{appendix}
%\input{data/appendix01}
%\end{appendix}

% 个人简历
%!Mode:: "TeX:UTF-8"
\begin{resume}
  \resumeitem{个人简历}
  1900 年 1 月出生于 C 省 D 市。

  2000 年 9 月考入 清华大学 电子系 电子信息科学与技术专业,3000 年 7 月本科毕业并获得工学学士学位, 同时获得免试攻读清华大学电子系博士学位资格。

  3000 年 9 月进入清华大学电子系攻读博士学位至今。

  \resumeitem{发表的学术论文} % 发表的和录用的合在一起

  \begin{enumerate}[{[}1{]}]
  \item M. Ni, and L. Mou, Some Paper Written by Me,
  IEEE International Conference on Acoustics, Speech, and Signal Processing  (ICASSP),
  0000-1111, May 4-9, 2014, Florence, Italy. (EI 收录, 检索号: 2014000000000.)
  \item M. Ni, and L. Mou, Some Paper Appeared in some Journal, IEEE Transactions on Signal Processing, 11(1):1111-2222, 2015.
   (SCI 收录, 检索号: WOS:000350000000020.)
  \end{enumerate}

  \resumeitem{研究成果} % 有就写,没有就删除

  \begin{enumerate}[{[}1{]}]
  \item 匿名君,某老师. 一种不知道是用来干什么的装置和方法,CN100000000B. (中国发明专利授权公告号.)
  \end{enumerate}

\end{resume}

\end{document}
